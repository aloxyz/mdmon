\documentclass[11pt]{article}
\usepackage[italian]{babel}
\usepackage[utf8x]{inputenc}
\usepackage{amsmath}
\usepackage{graphicx}
\usepackage[colorinlistoftodos]{todonotes}
\usepackage{enumitem}
\usepackage{listings}
\usepackage{filecontents}
\usepackage{verbatim}
\usepackage{eurosym}
\usepackage[export]{adjustbox}

\usepackage{float}
\usepackage[margin=3cm]{geometry}
\usepackage{listings}
\usepackage{xcolor}

\definecolor{codegreen}{rgb}{0,0.6,0}
\definecolor{codegray}{rgb}{0.5,0.5,0.5}
\definecolor{codepurple}{rgb}{0.58,0,0.82}
% \definecolor{backcolour}{rgb}{0.95,0.95,0.95}
\definecolor{backcolour}{rgb}{1,1,1}

\lstdefinestyle{mystyle}{
    backgroundcolor=\color{backcolour},   
    commentstyle=\color{codegreen},
    keywordstyle=\color{magenta},
    numberstyle=\tiny\color{codegray},
    stringstyle=\color{codepurple},
    basicstyle=\ttfamily\footnotesize,
    aboveskip=15pt, % Adjust the space above the listing
    belowskip=15pt, % Adjust the space below the listing
    breakatwhitespace=false,         
    breaklines=true,                 
    captionpos=b,                    
    keepspaces=true,                 
    numbers=left,                    
    numbersep=5pt,                  
    showspaces=false,                
    showstringspaces=false,
    showtabs=false,                  
    tabsize=2
}

\lstset{style=mystyle}

\begin{document}
\begin{titlepage}

    \newcommand{\HRule}{\rule{\linewidth}{0.5mm}} % Defines a new command for the horizontal lines, change thickness here

    \center % Center everything on the page

    %----------------------------------------------------------------------------------------
    %	HEADING SECTIONS
    %----------------------------------------------------------------------------------------

    \textsc{\LARGE Università di Messina}\\[1.5cm] % Name of your university/college
    \textsc{\Large Dipartimento di scienze matematiche e informatiche, scienze fisiche e della terra}\\[0.5cm] % Major heading such as course name
    \textsc{\large Corso di Laurea Triennale in Informatica}\\[0.5cm] % Minor heading such as course title

    %----------------------------------------------------------------------------------------
    %	TITLE SECTION
    %----------------------------------------------------------------------------------------

    \HRule \\[0.4cm]
    { \huge \bfseries "mdmon" - Monitoring delle Partizioni RAID}\\[0.4cm] % Title of your document
    \HRule \\[1.5cm]

    %----------------------------------------------------------------------------------------
    %	AUTHOR SECTION
    %----------------------------------------------------------------------------------------

    \begin{minipage}{0.4\textwidth}
        \begin{flushleft} \large
            \emph{Author:}\\
            Gabriele \textsc{Aloisio} \textit{(503264)} \\
        \end{flushleft}
    \end{minipage}
    ~
    \begin{minipage}{0.4\textwidth}
        \begin{flushright} \large
            \emph{Supervisor:} \\
            prof.ssa Maria Teresa \textsc{Reggio} \\
        \end{flushright}
    \end{minipage}\\[2cm]

    % If we don't want a supervisor, uncomment the two lines below and remove the section above
    %\Large \emph{Author:}\\
    %John \textsc{Smith}\\[3cm] % your name

    %----------------------------------------------------------------------------------------
    %	DATE SECTION
    %----------------------------------------------------------------------------------------

    {\large \today}\\[2cm] % Date, change the \today to a set date if we want to be precise

    %----------------------------------------------------------------------------------------
    %	LOGO SECTION
    %----------------------------------------------------------------------------------------

    \includegraphics[width=70px, keepaspectratio]{"unime.png"}\\[1cm] % Include a department/university logo - this will require the graphicx package

    %----------------------------------------------------------------------------------------

    \vfill % Fill the rest of the page with whitespace

\end{titlepage}

\tableofcontents
\pagebreak

\section{Abstract}
Questo progetto focalizza l'implementazione di un sistema \textbf{RAID} su un ambiente operativo \textit{Linux}. In particolare, il sistema è configurato con due dischi in mirroring (\textbf{RAID 1}) per garantire ridondanza e resilienza dei dati. Una partizione accessibile dal sistema operativo, configurata in \textbf{RAID 5}, ospita cartelle condivise attraverso \textbf{SAMBA}. Gli utenti hanno accesso esclusivamente alle cartelle condivise, mantenendo la separazione dall'accesso al sistema. Un sistema di alert è stato implementato per monitorare entrambi i livelli \textbf{RAID}, notificando gli amministratori di sistema tramite email in caso di guasto di un disco o altre anomalie. Lo script "\texttt{mdmon}", eseguito ogni ora tramite \texttt{cron}, gestisce il monitoraggio dello stato del \textbf{RAID}, inviando notifiche di eventi critici agli amministratori tramite il server \textbf{SMTP} di \textbf{Google}. I dettagli degli eventi sono registrati nel file di log \texttt{/var/log/mdmon.log} ad ogni esecuzione dello script. La directory \texttt{/share/} è limitata all'accesso solo da parte degli utenti del gruppo "\texttt{sambashare}" e opera su una partizione \textbf{RAID 5}, garantendo una gestione sicura e condivisa delle risorse.


\section{Introduzione}
Il nostro progetto si focalizza sulla progettazione e implementazione di una solida infrastruttura di storage basata su RAID in un ambiente operativo Linux. La configurazione di tale sistema prevede l'utilizzo di due dischi in mirroring, implementando la tecnologia RAID 1, al fine di garantire una ridondanza e una resilienza elevata dei dati. Attraverso la creazione di una partizione accessibile dal sistema operativo, configurata in RAID 5, siamo in grado di ospitare cartelle condivise mediante l'utilizzo di SAMBA. Un aspetto cruciale di questa configurazione è la gestione degli accessi, consentendo agli utenti di accedere esclusivamente alle cartelle condivise e mantenendo una separazione netta dall'accesso al sistema. Inoltre, al fine di garantire una pronta risposta a eventuali guasti dei dischi o a anomalie di sistema, abbiamo implementato un sistema di alert. Questo sistema avverte gli amministratori di sistema tramite e-mail in caso di malfunzionamenti, appoggiandosi a un server con autenticazione per garantire la sicurezza e l'affidabilità delle comunicazioni. Attraverso questa soluzione integrata, miriamo a garantire un ambiente di storage robusto, sicuro e facilmente gestibile.

\subsection{SAMBA}
SAMBA è una potente suite di software che facilita l'integrazione di sistemi basati su Linux e Windows in una rete. Con \textbf{SAMBA}, è possibile condividere file e risorse in modo fluido tra piattaforme eterogenee. La sua flessibilità e affidabilità lo rendono uno strumento essenziale per la creazione di ambienti collaborativi e condivisione di file in contesti aziendali e domestici. Grazie alle sue caratteristiche avanzate, \textit{SAMBA} permette agli utenti di accedere a cartelle condivise, stampanti e altri servizi, garantendo al contempo sicurezza e controllo degli accessi. La sua continua evoluzione e la comunità attiva lo rendono una scelta preferita per implementazioni di reti miste.

\subsection{RAID}
Il RAID, acronimo di \textit{Redundant Array of Independent Disks}, è una tecnologia di archiviazione che combina più dischi rigidi in un'unica unità logica. L'obiettivo principale è migliorare la prestazione e/o fornire ridondanza dei dati per aumentare l'affidabilità e la sicurezza del sistema di storage.
\\
Nel contesto specifico del nostro progetto, implementeremo due livelli di RAID: RAID 1 e RAID 5.

\begin{itemize}
    \item \textbf{RAID 1 (Mirroring):} In una configurazione RAID 1, due dischi rigidi contenenti gli stessi dati sono utilizzati in parallelo. Ogni dato scritto su un disco viene duplicato sull'altro, creando una copia identica. Questo livello di RAID offre un'elevata ridondanza, in quanto il sistema può continuare a operare senza interruzioni anche in caso di guasto di uno dei dischi. L'utilizzo di RAID 1 è particolarmente indicato per garantire l'integrità dei dati e la continuità operativa.

    \item \textbf{RAID 5:} Nel caso di RAID 5, la ridondanza dei dati viene ottenuta mediante la distribuzione delle informazioni di parità su tutti i dischi del RAID array. Questo schema permette al sistema di recuperare i dati in caso di guasto di uno dei dischi. Questo livello di raid è noto per bilanciare efficacemente le prestazioni e la ridondanza, risultando adatto per applicazioni che richiedono un buon compromesso tra capacità di storage, velocità di accesso ai dati e sicurezza.
\end{itemize}
L'utilizzo combinato di RAID 1 e RAID 5 nel nostro progetto mira a creare un ambiente di storage resiliente e performante, in grado di garantire l'integrità dei dati e la continuità operativa anche in situazioni critiche.

\subsection{Debian}
Debian è un sistema operativo open-source basato su Linux, rinomato per la sua stabilità, affidabilità e flessibilità. Essendo una distribuzione completamente gratuita, Debian è supportato da una vasta comunità di sviluppatori e utenti appassionati. La sua architettura basata su software libero e il rigoroso processo di selezione dei pacchetti garantiscono un ambiente sicuro e affidabile. Debian è una scelta popolare per una varietà di utilizzi, dalle implementazioni di server aziendali alle soluzioni personalizzate. Grazie al suo sistema di gestione dei pacchetti avanzato (APT), gli utenti possono facilmente installare, aggiornare e gestire le applicazioni con pochi comandi. Debian è particolarmente adatto per coloro che cercano un sistema operativo stabile e altamente personalizzabile.
\\\\
Per il nostro progetto, abbiamo selezionato \textbf{Debian 12 "Bookworm"} come sistema operativo Per diversi motivi:
\begin{itemize}
    \item Solida reputazione e vasta comunità di supporto, rendendolo un'ottima scelta per implementazioni di sistemi critici.
    \item Stabilità e facilità di utilizzo.
    \item La vasta raccolta di pacchetti garantiscono flessibilità nell'implementare diverse soluzioni.
\end{itemize}
Come ambiente desktop, abbiamo optato per \textbf{XFCE} per la sua leggerezza e semplicità. \textbf{XFCE} offre un'esperienza utente intuitiva senza compromettere le risorse di sistema, il che è cruciale per garantire prestazioni ottimali in ambienti server. La combinazione di \textbf{Debian} e \textbf{XFCE} ci fornisce un sistema robusto, facile da gestire e ottimizzato per le nostre esigenze di progetto.


\section{Caso di studio e implementazione}
Il progetto in questione presenta i seguenti quesiti:
\begin{enumerate}
    \item Si ha un sistema operativo (Linux) che gira su 2 dischi in mirroring (RAID 1)
    \item Si ha una partizione accessibile da S.O. (punto 1) in RAID 5 che contiene delle cartelle condivise (con SAMBA), l'utenza può accedere solo alle cartelle ma non al sistema (utenza solo samba).
    \item Sistema di Alert: per entrambi i RAID (RAID 1 del punto (1) e RAID 5 del punto(2)) nel caso di guasto di uno dei dischi o altra possibile anomalia il sistema deve inviare una e-mail agli amministratori di sistema
appoggiandosi a un server con autenticazione.
\end{enumerate}

\subsection{Installazione del sistema operativo e RAID}
\subsection{Configurazione di RAID 5}
\subsection{Creazione e configurazione dell'utenza}
\subsection{Installazione e configurazione di SAMBA}
\subsection{Sistema di alert}



\section{Conclusioni}


\end{document}