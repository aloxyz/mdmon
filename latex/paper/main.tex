\documentclass[11pt]{article}
\usepackage[italian]{babel}
\usepackage[utf8x]{inputenc}
\usepackage{amsmath}
\usepackage{graphicx}
\usepackage[colorinlistoftodos]{todonotes}
\usepackage{enumitem}
\usepackage{listings}
\usepackage{filecontents}
\usepackage{verbatim}
\usepackage{eurosym}
\usepackage[export]{adjustbox}

\usepackage{float}
\usepackage[margin=3cm]{geometry}
\usepackage{listings}
\usepackage{xcolor}

\definecolor{codegreen}{rgb}{0,0.6,0}
\definecolor{codegray}{rgb}{0.5,0.5,0.5}
\definecolor{codepurple}{rgb}{0.58,0,0.82}
% \definecolor{backcolour}{rgb}{0.95,0.95,0.95}
\definecolor{backcolour}{rgb}{1,1,1}

\lstdefinestyle{mystyle}{
    backgroundcolor=\color{backcolour},   
    commentstyle=\color{codegreen},
    keywordstyle=\color{magenta},
    numberstyle=\tiny\color{codegray},
    stringstyle=\color{codepurple},
    basicstyle=\ttfamily\footnotesize,
    aboveskip=15pt, % Adjust the space above the listing
    belowskip=15pt, % Adjust the space below the listing
    breakatwhitespace=false,         
    breaklines=true,                 
    captionpos=b,                    
    keepspaces=true,                 
    numbers=left,                    
    numbersep=5pt,                  
    showspaces=false,                
    showstringspaces=false,
    showtabs=false,                  
    tabsize=2
}

\lstset{style=mystyle}

\begin{document}
\begin{titlepage}

    \newcommand{\HRule}{\rule{\linewidth}{0.5mm}} % Defines a new command for the horizontal lines, change thickness here

    \center % Center everything on the page

    %----------------------------------------------------------------------------------------
    %	HEADING SECTIONS
    %----------------------------------------------------------------------------------------

    \textsc{\LARGE Università di Messina}\\[1.5cm] % Name of your university/college
    \textsc{\Large Dipartimento di scienze matematiche e informatiche, scienze fisiche e della terra}\\[0.5cm] % Major heading such as course name
    \textsc{\large Corso di Laurea Triennale in Informatica}\\[0.5cm] % Minor heading such as course title

    %----------------------------------------------------------------------------------------
    %	TITLE SECTION
    %----------------------------------------------------------------------------------------

    \HRule \\[0.4cm]
    { \huge \bfseries "mdmon" - Multiple Device Monitoring}\\[0.4cm] % Title of your document
    progetto di esame per Laboratorio di Amministrazione di Sistemi
    \HRule \\[1.5cm]

    %----------------------------------------------------------------------------------------
    %	AUTHOR SECTION
    %----------------------------------------------------------------------------------------

    \begin{minipage}{0.4\textwidth}
        \begin{flushleft} \large
            \emph{Author:}\\
            Gabriele \textsc{Aloisio} \textit{(503264)} \\
        \end{flushleft}
    \end{minipage}
    ~
    \begin{minipage}{0.4\textwidth}
        \begin{flushright} \large
            \emph{Supervisor:} \\
            prof.ssa Maria Teresa \textsc{Reggio} \\
        \end{flushright}
    \end{minipage}\\[2cm]

    % If we don't want a supervisor, uncomment the two lines below and remove the section above
    %\Large \emph{Author:}\\
    %John \textsc{Smith}\\[3cm] % your name

    %----------------------------------------------------------------------------------------
    %	DATE SECTION
    %----------------------------------------------------------------------------------------

    {\large \today}\\[2cm] % Date, change the \today to a set date if we want to be precise

    %----------------------------------------------------------------------------------------
    %	LOGO SECTION
    %----------------------------------------------------------------------------------------

    \includegraphics[width=70px, keepaspectratio]{"unime.png"}\\[1cm] % Include a department/university logo - this will require the graphicx package

    %----------------------------------------------------------------------------------------

    \vfill % Fill the rest of the page with whitespace

\end{titlepage}

\tableofcontents
\pagebreak

\section{Abstract}
Questo progetto tratta l'implementazione di un sistema \textbf{RAID} su un ambiente operativo \textbf{Linux}. Il sistema è configurato con due dischi in mirroring (\textbf{RAID 1}) per permettere ridondanza e resilienza dei dati. Una partizione accessibile dal sistema operativo, configurata in \textbf{RAID 5}, ospita cartelle condivise attraverso \textbf{SAMBA}. Gli utenti hanno accesso esclusivamente alle cartelle condivise, mantenendo la separazione dall'accesso al sistema. Un sistema di alert è stato implementato per monitorare entrambi i livelli \textbf{RAID}, notificando gli amministratori di sistema tramite email in caso di guasto di un disco o altre anomalie. Lo script "\texttt{mdmon}", eseguito ogni ora tramite \texttt{cron}, gestisce il monitoraggio dello stato del \textbf{RAID}, inviando notifiche di eventi critici agli amministratori tramite il server \textbf{SMTP} di \textbf{Google}. I dettagli degli eventi sono registrati nel file di log \texttt{/var/log/mdmon.log} ad ogni esecuzione dello script. La directory \texttt{/share/} è limitata all'accesso solo da parte degli utenti del gruppo "\texttt{sambashare}" e opera su una partizione \textbf{RAID 5}, garantendo una gestione sicura e condivisa delle risorse.

\pagebreak

\section{Introduzione}
Il nostro progetto si focalizza sulla progettazione e implementazione di un'infrastruttura di storage basata su RAID in un ambiente operativo Linux. La configurazione di tale sistema prevede l'utilizzo di due dischi in mirroring, implementando la tecnologia \textbf{RAID 1}, al fine di permettere ridondanza e resilienza dei dati. Con la creazione di una partizione accessibile dal sistema operativo, configurata in \textbf{RAID 5}, siamo in grado di ospitare cartelle condivise mediante \textbf{SAMBA}. Viene successivamente configurata la gestione degli accessi, consentendo agli utenti di accedere esclusivamente alle cartelle condivise. In fine, al fine di garantire una pronta risposta a eventuali guasti dei dischi, abbiamo implementato un sistema di alert. Questo sistema avverte gli amministratori di sistema tramite e-mail in caso di malfunzionamenti, appoggiandosi a un server con autenticazione per garantire la sicurezza e l'affidabilità delle comunicazioni. Il progetto verrà fatto eseguire su una macchina virtuale utilizzando \textbf{KVM} (\textit{Kernel-based Virtual Machine}), una tecnologia di virtualizzazione open source integrata nel kernel Linux.

\subsection{SAMBA}
SAMBA è una suite di software che facilita l'integrazione di sistemi basati su Linux e Windows in una rete. Con \textbf{SAMBA}, è possibile condividere file e risorse tra piattaforme eterogenee. \textit{SAMBA} permette agli utenti di accedere a cartelle condivise, stampanti e altri servizi, con sicurezza e controllo degli accessi.

\subsection{RAID}
Il RAID, acronimo di \textit{Redundant Array of Independent Disks}, è una tecnologia di archiviazione che combina più dischi rigidi in un'unica unità logica. L'obiettivo principale è migliorare la prestazione e/o fornire ridondanza dei dati per aumentare l'affidabilità e la sicurezza del sistema di storage.
\\
Nel contesto specifico del nostro progetto, implementeremo due livelli di RAID: \textbf{RAID 1} e \textbf{RAID 5}.

\begin{itemize}
    \item \textbf{RAID 1 (Mirroring):} In una configurazione RAID 1, due dischi rigidi contenenti gli stessi dati sono utilizzati in parallelo. Ogni dato scritto su un disco viene duplicato sull'altro, creando una copia identica. Questo livello di RAID offre un'elevata ridondanza, in quanto il sistema può continuare a operare senza interruzioni anche in caso di guasto di uno dei dischi.

    \item \textbf{RAID 5:} Nel caso di RAID 5, la ridondanza dei dati viene ottenuta mediante la distribuzione delle informazioni di parità su tutti i dischi del RAID array. Questo schema permette al sistema di recuperare i dati in caso di guasto di uno dei dischi. Risulta adatto per bilanciare efficacemente le prestazioni e la ridondanza.
\end{itemize}

\pagebreak

\subsection{Debian}
Debian è un sistema operativo open-source basato su Linux, rinomato per la sua stabilità, affidabilità e flessibilità, ed è supportato da una vasta comunità di sviluppatori e utenti appassionati. La sua architettura basata su software libero e il rigoroso processo di selezione dei pacchetti garantiscono un ambiente sicuro e affidabile. Debian è una scelta popolare per una varietà di utilizzi, dalle implementazioni di server aziendali alle soluzioni personalizzate. Grazie al suo \textit{package manager} \textbf{APT}, gli utenti possono facilmente installare, aggiornare e gestire le applicazioni con pochi comandi. Debian è particolarmente adatto per coloro che cercano un sistema operativo stabile e altamente personalizzabile.
\\\\
Per il nostro progetto, abbiamo selezionato \textbf{Debian 12 "Bookworm"} come sistema operativo Per diversi motivi:
\begin{itemize}
    \item Solida reputazione e vasta comunità di supporto, rendendolo un'ottima scelta per implementazioni di sistemi critici.
    \item Stabilità e facilità di utilizzo.
    \item La vasta raccolta di pacchetti garantiscono flessibilità nell'implementare diverse soluzioni.
\end{itemize}
Come ambiente desktop, abbiamo optato per \textbf{XFCE} per la sua leggerezza e semplicità. \textbf{XFCE} offre un'esperienza utente intuitiva senza compromettere le risorse di sistema, il che è importante per garantire prestazioni ottimali in ambienti server. La combinazione di \textbf{Debian} e \textbf{XFCE} ci fornisce un sistema stabile, facile da gestire e ottimizzato per le nostre esigenze di progetto.

\subsection{Postfix}
Postfix è un server di trasferimento di posta (MTA) utilizzato per la gestione dell'invio e della ricezione delle email in un sistema Linux. Nel contesto del nostro progetto, utilizzeremo Postfix per l'autenticazione e il relay delle email tramite il server SMTP di Google. Ciò significa che Postfix fungerà da intermediario tra il nostro sistema e il server SMTP di Google per inviare le email. L'autenticazione è importante per garantire che solo utenti autorizzati possano inviare email attraverso il nostro sistema.

\subsection{Cron}
Cron è un servizio in ambiente Unix e Unix-like che consente agli utenti di programmare l'esecuzione automatica di comandi o script a intervalli specifici di tempo, giorni della settimana o mesi. In merito al progetto, utilizzeremo cron per automatizzare l'esecuzione periodica dello script \textbf{mdmon}. Questo script è progettato per monitorare lo stato dei dischi in un array RAID e inviare notifiche agli amministratori in caso di anomalie o malfunzionamenti.

\pagebreak

\section{Caso di studio e implementazione}
Il progetto in questione presenta i seguenti quesiti:
\begin{enumerate}
    \item Si ha un sistema operativo (Linux) che gira su 2 dischi in mirroring (RAID 1)
    \item Si ha una partizione accessibile da S.O. (punto 1) in RAID 5 che contiene delle cartelle condivise (con SAMBA), l'utenza può accedere solo alle cartelle ma non al sistema (utenza solo samba).
    \item Sistema di Alert: per entrambi i RAID (RAID 1 del punto (1) e RAID 5 del punto(2)) nel caso di guasto di uno dei dischi o altra possibile anomalia il sistema deve inviare una e-mail agli amministratori di sistema
appoggiandosi a un server con autenticazione.
\end{enumerate}

\subsection{Configurazione della macchina virtuale}
La macchina, su cui viene installato Debian 12.1.0, è stata configurata nel modo seguente:
\begin{itemize}
    \item 4 vCPU
    \item RAM 4GB
    \item 2 dischi di archiviazione da 16GB
    \item 3 dischi di archiviazione da 1GB
\end{itemize}

\subsection{Installazione del sistema operativo}
L'installazione grafica di Debian fornisce un'interfaccia utente intuitiva che semplifica la configurazione del sistema operativo durante il processo di installazione. Questa modalità di installazione permette di definire varie configurazioni. Di seguito sono elencati alcuni dei parametri:
\begin{itemize}
    \item Hostname: L'hostname è il nome univoco assegnato al computer nella rete.
    \item Username: Durante la procedura di installazione, è possibile definire il nome utente principale per l'account amministrativo. Questo sarà l'account con privilegi di superutente (root).
    \item Password: Si può impostare la password associata all'account root e agli altri account utente creati durante l'installazione.
    \item Zona Oraria: L'utente può selezionare la zona oraria appropriata per il sistema, così che l'orologio del sistema sia correttamente sincronizzato con il fuso orario locale.
    \item Selezione del Software: Durante l'installazione grafica, è possibile scegliere quali gruppi di pacchetti e software installare. Quì viene scelto di installare XFCE come desktop environment (DE).
    \item Rete: Configurazione delle impostazioni di rete, tra cui la configurazione manuale o automatica dell'indirizzo IP, la configurazione del gateway, e altre opzioni relative alla connettività di rete.
    \item Partizionamento del Disco: Gli utenti hanno la possibilità di definire la configurazione delle partizioni del disco rigido.
\end{itemize}
Avremo un partizionamento di questo tipo:
\begin{figure}[H]
    \includegraphics[width=0.8\textwidth, keepaspectratio]{../img/lsblk.png}
    \centering
\end{figure}

dove:
\begin{itemize}
    \item \texttt{vda} e \texttt{vdb} sono i dischi per il sistema operativo
    \item \texttt{vdc}, \texttt{vdd} e \texttt{vde} sono i dischi per lo sharing con SAMBA
    \item \texttt{md0} e \texttt{md1} sono i dispositivi \textit{multiple device}, rispettivamente di tipo RAID1 e RAID5
    \item \texttt{vda5} e \texttt{vdb5} è la partizione di SWAP
    \item \texttt{vda2} e \texttt{vdb2} è una partizione di tipo \texttt{udev}    
\end{itemize}

\pagebreak

\subsubsection{Configurazione della partizione RAID 1}
In questo caso l'installer ci permette comodamente di configurare la partizione \textbf{RAID 1} per il mountpoint di root (/) prima ancora di installare Debian. Il processo utilizzato è il seguente:

\begin{enumerate}
    \item Creare due partizioni vuote per \texttt{vda} e \texttt{vdb}
    \item Impostare il tipo delle partizioni come \textit{volume fisico per RAID}
    \item Creare un \texttt{MD} (multiple device) di tipo RAID1 (\texttt{md0})
    \item Impostare il numero di device attivi a 2
    \item Impostare il numero di spare device a 0
    \item Selezionare i device da utilizzare per l'array RAID1 (\texttt{vda1} e \texttt{vdb1})
    \item Creare Il filesystem di root sulla partizione RAID1 appena creata
\end{enumerate}


\subsection{Configurazione di RAID 5}
Installiamo \texttt{mdadm} con
\begin{verbatim}
$ sudo apt update
$ sudo apt install mdadm -y
\end{verbatim}
Sucecssivamente formattiamo e partizioniamo i dischi \texttt{vdc}, \texttt{vdd} e \texttt{vde} con \texttt{fdisk}:
\begin{lstlisting}
$ fdisk /dev/vdc

Welcome to fdisk (util-linux 2.34).
Changes will remain in memory only, until you decide to write them.
Be careful before using the write command.

The old ext4 signature will be removed by a write command.

Device does not contain a recognized partition table.
Created a new DOS disklabel with disk identifier 0x26a2d2f6.

Command (m for help): n
Partition type
   p   primary (0 primary, 0 extended, 4 free)
   e   extended (container for logical partitions)
Select (default p): p
Partition number (1-4, default 1): 1
First sector (2048-20971519, default 2048):
Last sector, +/-sectors or +/-size{K,M,G,T,P} (2048-2097151, default 2097151):

Created a new partition 1 of type 'Linux' and of size 1 GiB.

Command (m for help): t
Selected partition 1
Hex code (type L to list all codes): fd
Changed type of partition 'Linux' to 'Linux raid autodetect'.

Command (m for help): w
The partition table has been altered.
Calling ioctl() to re-read partition table.
Syncing disks.
\end{lstlisting}
Creiamo l'array RAID5
\begin{verbatim}
$ mdadm --create /dev/md1 --level=5 --raid-devices=3 /dev/vdc /dev/vdd /dev/vde
\end{verbatim}
Ed un filesystem \texttt{ext4}
\begin{verbatim}
$ mkfs.ext4 /dev/md1
\end{verbatim}
Lo montiamo con
\begin{verbatim}
$ mount /dev/md1 /share
\end{verbatim}
E rendiamo la configurazione persistente inserendo questa voce in \texttt{/etc/fstab}
\begin{verbatim}
/dev/md1                /share     ext4   defaults                0 0
\end{verbatim}
Per default, RAID non dispone di un file di configurazione e quindi dobbiamo salvarlo manualmente. Se questo passaggio non viene seguito, il dispositivo RAID potrebbe essere denominato in un modo diverso da \texttt{md1}. Quindi, è necessario salvare la configurazione affinché persista tra i riavvii; quando avviene un riavvio, viene caricata nel kernel e il RAID viene anch'esso caricato:

\begin{verbatim}
$ mdadm --detail --scan --verbose >> /etc/mdadm.conf
\end{verbatim}

\subsection{Installazione e configurazione di SAMBA}
Installiamo SAMBA
\begin{verbatim}
$ sudo apt install samba
\end{verbatim}
E aggiungiamo il mountpoint \texttt{/share} in \texttt{/etc/samba/smb.conf}, specificando che gli utenti facenti parte del gruppo \texttt{sambashare} possono accedere alla directory:
\begin{lstlisting}
[share]
    comment = Samba shared directory
    path = /share
    read only = no
    browsable = yes
    valid users = @sambashare
\end{lstlisting}
Concludiamo riavviando il servizio
\begin{verbatim}
$ sudo systemctl restart smbd.service
\end{verbatim}

\subsection{Creazione e configurazione dell'utenza}
Per dimostrare il funzionamento dei permessi della shared directory, creiamo due utenti \textit{user1} e \textit{user2}, dove solo \textit{user1} farà parte del gruppo \textit{sambashare}:
\begin{verbatim}
$ sudo useradd -m user1
$ sudo useradd -m user2
$ sudo groupadd sambashare
$ sudo usermod -aG sambashare user1
\end{verbatim}
Avremo quindi
\begin{verbatim}
$ groups user1
user1 : user1 users sambashare
$ groups user2
user2 : user2 users
\end{verbatim}

\pagebreak

\subsection{Sistema di alert}
Lo script vero e proprio \texttt{/etc/cron.hourly/mdmon} viene eseguito ogni ora da \textit{cron}. Questo script legge un altro file di testo \texttt{/home/admin/fault\_msg} dove scriveremo il messaggio che \textbf{mdmon} invierà al destinatario. Un file di log \texttt{/var/log/mdmon.log} viene creato, dove si tiene traccia dei \textit{check} che esegue \textit{mdmon}. Il \textit{check} funziona controllando l'output del comando \texttt{mdadm --detail <dispositivo md>}, controllando con \texttt{awk} se lo stato è \textbf{degraded}. In questo caso, viene subito inviata una mail con il comando \texttt{mail} di \texttt{gnumail}.

\begin{lstlisting}[language=Bash, caption={checkmd.sh}]
#!/bin/bash

log_file="/var/log/mdmon.log"
fault_msg="/home/admin/fault_msg"
recipient="example@mail.com"
    
echo "$(date):" >> $log_file
    
for i in /dev/md/*
do
    state=$(sudo mdadm --detail $i | grep State | head -n 1 | awk '{print $NF}')
    if [[ $state = "degraded" ]]
    then
        mail -s "Degraded RAID partition ${i}" $recipient < $fault_msg
        echo "${i} faulty. Sent notification to administrator." >> $log_file
    else
        echo "${i} ok." >> $log_file
    fi
done
    
echo "" >> $log_file
    
\end{lstlisting}

\begin{lstlisting}[caption={fault\_msg}]
This is an automated notification.

Dear Administrator,
This message was sent because one of the RAID partitions is in a degraded state.
It is suggested to intervene.
    
Sincerely,
The automated notification system
\end{lstlisting}

\subsubsection{Configurazione di Postfix}
Installiamo Postfix con
\begin{verbatim}
$ sudo apt install mailutils
\end{verbatim}
e seguiamo la configurazione del servizio, specificando il \textit{FQDN} (Fully Qualified Domain Name) come l'hostname della nostra macchina.
\\
\\
Configuriamo Postfix aggiungendo delle opzioni per l'autenticazione e il nostro relayhost al file \texttt{/etc/postfix/main.cf}:
\begin{lstlisting}[caption={main.cf}]
smtpd_relay_restrictions = permit_mynetworks permit_sasl_authenticated defer_unauth_destination
myhostname = labbox.org
alias_maps = hash:/etc/aliases
alias_database = hash:/etc/aliases
myorigin = /etc/mailname
mydestination = localhost.labbox.org, , localhost
relayhost = [smtp.gmail.com]:587
mynetworks = 127.0.0.0/8 [::ffff:127.0.0.0]/104 [::1]/128
mailbox_size_limit = 0
recipient_delimiter = +
inet_interfaces = all
inet_protocols = all
    
smtp_sasl_auth_enable = yes
smtp_sasl_security_options = noanonymous
smtp_sasl_password_maps = hash:/etc/postfix/sasl/sasl_passwd
smtp_tls_security_level = encrypt
smtp_tls_CAfile = /etc/ssl/certs/ca-certificates.crt
\end{lstlisting}
Creiamo il file delle credenziali \texttt{/etc/postfix/sasl/sasl\_passwd} e scriviamo dentro
\begin{verbatim}
[smtp.gmail.com]:587 example@email.com:<tua password>
\end{verbatim}
Con \texttt{postmap} possiamo convertire il file appena creato in un database
\begin{verbatim}
$ sudo postmap /etc/postfix/sasl/sasl_passwd
\end{verbatim}
Questo creerà un file \texttt{sasl\_passwd.db} nella stessa directory.
\\
Cambiamo i permessi del file per permettere solo all'utente \textit{root} di leggere e scrivere
\begin{verbatim}
$ sudo chown root:root /etc/postfix/sasl/sasl_passwd
$ sudo chmod 600 /etc/postfix/sasl/sasl_passwd
\end{verbatim}
Infine riavviamo il servizio Postfix
\begin{verbatim}
$ sudo systemctl restart postfix
\end{verbatim}

Bisognerà poi accedere al \textbf{pannello di controllo dell'account Google} per permettere ad "app meno sicure" di eseguire l'accesso.

\section{Conclusioni}


\end{document}